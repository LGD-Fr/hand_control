\documentclass[fontsize=12pt, DIV=calc, a4paper]{scrartcl}
%\documentclass[12pt, DIV=calc, twosides=on, a4paper,twocolumn]{scrartcl}

\usepackage{fontspec}
\usepackage{polyglossia}
\usepackage{hyperref}
\usepackage[modulo]{lineno}
\usepackage{url}

\setdefaultlanguage{french}
%\setmainfont{Junicode}
%\setsansfont{Junicode}


%\setmainfont{EB Garamond}
%\setsansfont{EB Garamond}
%\newfontface\smallcaps[RawFeature={+c2sc,+scmp}]{EB Garamond}

%\setmainfont[Ligatures=TeX]{Linux Libertine O}
%\setsansfont{Linux Biolinum O}
%\newfontface\sc[Letters=SmallCaps]{Linux Biolinum O}

\title{\textsc{Hand Control}\\
 \textbf{Bibliographie}}

\author{Luc Absil\\
\href{mailto:luc.absil@supelec.fr}{luc.absil@supelec.fr}\\
~\\
 Louis-Guillaume Dubois\\
 \href{mailto:louis-guillaume.dubois@supelec.fr}{louis-guillaume.dubois@supelec.fr}\\
 ~\\
Paul Janin\\
\href{mailto:paul.janin@supelec.fr}{paul.janin@supelec.fr}
}
\date{}

\KOMAoptions{DIV=last}
\KOMAoptions{DIV=10}

% \figw{nomDuFichier.eps}{taille}{label}{ Mon beau titre.}
\newcommand\figw[4
]
{
	\begin{figure}[p]
	\centering
	\caption{\label{#3}  #4}
	\includegraphics[#2]{#1}
	\end{figure}
}

% \fig{nomDuFichier.eps}{label}{ Mon beau titre.}
\newcommand\fig[3]
{
	\begin{figure}[h]
	\centering


	\caption{\label{#2}  #3}
	\includegraphics[width=15cm]{#1}
	\end{figure}
}
\bibliographystyle{plain}
% pour les références
\newcommand\rr[1]{\ref{#1} page~\pageref{#1}}
\begin{document}
\maketitle

\tableofcontents

\section{Présentation}
\subsection{Introduction}
Cette synthèse documentaire s'intéresse à la plateforme de développement ROS pour le contrôle de systèmes automatisés, ainsi qu'au dispositif de détection de mouvement Kinect, développé par Microsoft, et à son utilisation dans le cadre de notre projet de synthèse. On s'attachera notamment aux divers cadres d'utilisations possibles pour le Kinect.

\subsection{Présentation de la recherche documentaire}
Notre sujet de synthèse portant sur des équipements récents (une kinect), aucun livre n'était répertorié sur le catalogue en ligne de la bibliothèque tricampus de Supélec. 

Nous avons utilisé différents outils externes pour établir la bibliographie présentée dans ce document.

\section{Inspection des bases de données}
Nous avons utilisé la base de donnée \emph{Inspec} pour trouver les articles \cite{performance} \cite{Kinect-3D} \cite{Kinect-robotic}. Toutefois, les articles qui nous intéressaient n'étaient pas disponibles en accès libre. Sur la base de donnée \emph{Science Direct}, nous avons trouvé \cite{Alisher}.

\section{Recherche des documentations des logiciels}
Grâce à un moteur de recherche usuel, tel \emph{DuckDuckGo}, il nous a été facile de trouver la documentation officielle de ROS\cite{Tutoriels}, sur laquelle une page était consacrée aux drivers nécessaires\cite{ROS} pour utiliser une kinect, ainsi que nous projetons de le faire dans notre projet.

Toujours en utilisant \emph{DuckDuckGo}, avec les mots clés « \emph{Kinect} » et « Microsoft », nous avons trouvé le site officiel de Microsoft sur le développement des kinects\cite{Kinectwindows}, ainsi que la documentation du projet \emph{OpenKinect} qui développe un pilote libre \emph{libfreenect}\cite{OpenKinect}, hébergé sur GitHub.


\bibliography{biblio} 
\end{document}
